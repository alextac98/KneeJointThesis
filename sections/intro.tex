\chapter{Introduction}
Sudden walking disabilities, such as lower limb paralysis, change a person's mobility, and often have quite a significant effect on a person's lifestyle and health. Paraplegia can introduce a significant challenge to maintain healthy bone and muscle mass, as the patient usually will have a much harder time exercising. However, research has shown that controlled rehabilitation can actually help reconnect a patient's injured neurons, reducing the effects of the injury and regaining mobility and/or feeling in their lower limbs \cite{GaitTrainingClinical}. As such, a large part of a paralysis victim's life is devoted to rehabilitation and physical therapy.

Exercises for rehabilitation come in many forms, from stretching to strength training, to hydrotherapy. Gait training has specifically been shown to improve the quality of life of a lower-limb paralysis patient, but this exercise is very hard to do, especially with severe cases of paraplegia. Robotics and exoskeletons have been proposed and tested to help with this cause. A mechanical system with intelligent software control can be used to assist and hold a person's weight as they perform exercises as called for by their doctor. Such systems can decrease bone and muscle atrophy and help fight the side effects of paraplegia. 

Most rehabilitation exoskeletons have placed their focus on the software control, and assumed joints like the hip, knee, and ankle to be hip joints. However, the knee joint specifically is known to not be a pin joint, as it linearly throughout flexion. This tibiofemoral relationship is often ignored, which can cause skin complications due to poor fitting when exoskeleton usage expands from just small clinical trials. This thesis aims to design and test a biomechanical actuated knee orthosis that can be customized to the natural tibiofemoral motion of a patient. It also aims to identify the tibiofemoral relationship in a person using modern imaging systems, such as motion capture systems and magnetic resonance imaging (MRI).

% Remake all of this
Robotics have a very large opportunity to improve the rehabilitation process in paraplegic patients. Clinical research in using robotic orthoses for gait training and other rehabilitation exercises show positive improvement in most patients when considering quality of life \cite{GaitTrainingBenefitsRoboticsWalkbot} \cite{RoboticGaitTraining}. 