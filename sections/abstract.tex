\begin{abstract}
    %%%%%%%%%%%%%%%%%%%%%%%%%%%%
    % ---- Short Abstract ---- %
    %%%%%%%%%%%%%%%%%%%%%%%%%%%%

    % Current research demonstrates that the shank (lower leg) linearly extends as the knee flexes, in a roughly quartic trajectory. However, most actuated orthoses do not consider this tibiofemoral relationship. This thesis presents a customizable biomechanical orthotic knee joint for medical rehabilitation which can follow this relationship. The design is easily manufacturable with common machining tools and FDM 3D printing techniques. Most importantly, it is customizable to each patient with the simple replacement of one component. Finally, this thesis will present a software workflow and tools to identify this tibiofemoral relationship in patients using motion capture technologies. 

    %%%%%%%%%%%%%%%%%%%%%%%%%%%
    % ---- Long Abstract ---- %
    %%%%%%%%%%%%%%%%%%%%%%%%%%%

    Paraplegia - the loss of feeling and/or movement in one's lower limbs - affects hundred of thousands of people in the United States, and can be caused by spinal cord injury or diseases such as multiple sclerosis, stroke, and cancer. Paralysis patients can help alleviate side effects through physical therapy and rehabilitation. Gait training with exoskeletons may be particularly effective, with some clinical research reporting even fully paralyzed individuals reducing their side effects. However, orthoses used with these patients must be carefully designed to ensure safety and comfort and avoid rubbing between the orthosis and the patient's skin, as dermatological issues can often pose risks to these individuals due to the reduced sensation and blood flow to affected limbs.
    
    Current lower limb rehabilitation exoskeletons for gait training use pin joints for the 6 powered limbs (ankle, knee, and hip joints). However current research demonstrates that the shank (lower leg) linearly extends as the knee joint flexes. Tis tibiofemoral relationship is not usually considered by developers of lower limb orthoses.

    This thesis presents a customizable biomechanical knee orthosis for medical rehabilitation which follow a patient's specific knee trajectory. The design can be manufactured with common machining tools and FDM 3D printing techniques. It can also be customized to each patient with the replacement of a single component. The testing and analysis demonstrates that the proposed orthosis can match and exceed the requirements presented, and successfully matches a defined tibiofemoral relationship.

    Additionally, this thesis introduces a software workflow with motion capture systems to identify the knee joint parameters in a patient for customization of the proposed orthosis. The software developed was tested with pilot data to verify processes and improve the workflow presented. Finally, a clinical human trial is outlined to verify the effectiveness of the proposed orthosis, test the effectiveness of the software workflow, and identify the relationship between the knee joint and the surrounding skin using a mixture of motion capture data and magnetic resonance imaging.

\end{abstract}

