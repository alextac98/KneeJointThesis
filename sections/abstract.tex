\begin{abstract}
    % Short Abstract:
    Current research demonstrates that the shank (lower leg) gets longer as the knee flexes, in a roughly quartic trajectory. However, most actuated orthotics do not consider this tibiofemoral relationship. This thesis presents a customizable biomechanical orthotic knee joint for medical rehabilitation which can follow this relationship. The design is easily manufacturable with common machining tools and FDM 3D printing techniques. Most importantly, it is customizable to each patient with the simple replacement of one component. Finally, this thesis will present a software workflow and tools to identify this tibiofemoral relationship in patients using motion capture technologies.

    % Long Abstract
\end{abstract}

