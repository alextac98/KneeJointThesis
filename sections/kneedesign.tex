\chapter{Knee Joint Design}

The knee joint presented in this thesis is designed to replace the current passive pin knee joint with a spring wrap clutch for the WPI LARRE exoskeleton (see \autoref{sec:larre}).

\TODO{add more in the intro section for the knee joint design}

\section{Design Parameters}

\subsubsection{Follow a specific equation}

\begin{equation}
    r(\theta) mm = 1.078\theta^4 - 11.184\theta^3 + 26.524\theta^2 - 0.825\theta + 263.59
    \label{eq:KneeJointGeometryEquation}
\end{equation}

\subsubsection{Parameter 2}


\section{Mechanical Design}

The orthotic joint design proposed uses a similar idea to how a human knee joint works; a cam mechanism  extends the shank link as it is rotated relative to the thigh link. The joint therefore has two degrees of freedom: rotation around the center of rotation (output shaft of the motor and gearbox) and translation in the direction of the shank. However, since there is only one actuator, the joint is underactuated; this underactuation can be taken advantage of to match a patient's knee trajectory, where the center of mass of the shank extends away from the joint center as the joint bends.

\begin{figure} [ht!]
    \centering
    \missingfigure{Knee Joint Exploded View}
    \caption{Exploded view of the knee joint, with all relevant components labeled}
    \label{fig:KneeJointExplodedView}
\end{figure}

\subsubsection{Torsion Bars}
The center of rotation of the joint is designed to match the axis of rotation of the actuator. The output of this actuator \fix{Add reference to actuator decision section} is directly connected to the torsion bar using M5 shoulder bolts. Each bolt is designed to support 3 bearings: 2 on the motor side and 1 on the patient side. The reduced count on the patient side allows for the torsion bar to be partially recessed in the shank link to reduce the distance between the center of mass between the patient and the joint. The 6 bearings are still able to support the forces necessary throughout a walking gait cycle (see \autoref{sec:BearingsAndCalcs}). 

\subsubsection{Shank Links}
The 2 shank links attach to the lower part of the exoskeleton, and are responsible for taking the rotational energy created by the motor and partially changing it to translational energy to help linearly extend the shank. The bearings connected to the torsion bars ride in a guide built into the shank link. This guide is slightly larger than the bearing diameter (\(~0.3mm\)) to prevent rubbing without creating much of a backlash (\(0.39^\circ\) backlash, see calculation on \autoref{eq:ShankLinkBacklash}).

\begin{equation}
    Backlash = atan(\frac{\frac{0.3mm}{2}}{22mm}) = 0.39^\circ
    \label{eq:ShankLinkBacklash}
\end{equation}

The surface of the guide must be smooth and parallel to the axis of the bearings to avoid damaging them. Depending on the material and manufacturing method chosen, the surface may require additional machining to ensure it can match these requirements. The length of the guide must be larger than the distance between the centers of the two shoulder bolts plus the maximum distance of linear extension by the knee (\autoref{eq:ExtensionGuideLength}). For this prototype, this length was \(78mm\).

\begin{equation}
    GuideLength \geq TorsionBarC2C + MaxKneeExtension = 44mm + 34mm = 78mm
    \label{eq:ExtensionGuideLength}
\end{equation}

The shank link is also responsible to connect to the lower part of the exoskeleton. Just like the thigh link, this is done through the universal exoskeleton connector developed throughout the WPI LARRE project \cite{SpringWrapClutchKnee}.

The connection between the thigh link and the shank link is very important, as it adds torsional stability and overall rigidness to the entire joint. It was therefore imperative during the design process to create wide surface contact between the thigh and shank links. To reduce the energy lost to friction between these plates, 3.2mm thick Delrin\textsuperscript{\textregistered} slides were laser cut and attached to the shank link. 

Similarly to the torsion bar, the shank link also uses 2 shoulder bolts to clamp the two shank links on the thigh link as well as to give the bearings that ride on the knee path guide a precise surface to mount to. To maintain a consistent clamping force, lock nuts are used since they do not easily back out with movement and vibration. 

\subsubsection{Thigh Link}

\begin{figure}
    \centering
    \missingfigure[]{Add cross section of the joint}
    \caption{A cross section of the knee joint in a \(0^\circ\) position}
    \label{fig:KneeJointCrossSection}
\end{figure}

The thigh link acts as the main mounting point for most things, as well as contains the knee path guide. Just like the shank link, the thigh link has the universal exoskeleton connector used throughout the WPI LARRE project. The motor bracket is connected to the thigh link at two locations using \(20mm\diameter x 50mm\) spacers. These spacers must be strong and stiff, as they transmit the torque between the thigh and shank connector in high load situations. A potentiometer is also mounted inside the thigh link to measure the current angle of the joint, as shown in \autoref{fig:KneeJointCrossSection}. The wire connecting to it is routed through a slot in the thigh link to avoid any interference with the moving shank links. This wire comes out the top and is connected to the main controller of the exoskeleton.

\subsubsection{Knee Path Guide}
The knee path guide is built into the thigh link as a slot. The geometry is calculated using several point measurements connected in SolidWorks with a spline. Each point is split by \TODO{Verify this number}{15 degrees}, and calculated from a pre-determined equation. This equation can be measured from a patient knee (see \autoref{sec:KneeParams}), but throughout the design and testing of this knee joint, \autoref{eq:KneeJointGeometryEquation} from \cite{KinDynKneeJoint} is used. \autoref{fig:CenterPlateGeometry} shows the equation above overlayed on the thigh link.

\begin{figure}[ht!]
    \centering
    \missingfigure{Make a picture of the thigh link with relationship highlighted}
    \caption{The thigh link contains the geometry which the bearings ride on to mimic the tibiofemoral relationship}
    \label{fig:CenterPlateGeometry}
\end{figure}

The joint is designed to be easily adaptable between patients. Therefore, the only customized part in the entire system is the thigh link which holds the knee path guide. All other parts remain the same to decrease cost and improve repairability.

\subsubsection{Torque Requirements \& Actuator Selection}
\TODO[inline]{Talk about torque requirement}
\TODO[inline]{Talk about drive train: motor, gearbox selection and output numbers}

\subsubsection{Potentiometer}
\TODO[inline]{Talk about the pot and its purpose}

\subsubsection{Motor Analog} 
\TODO{change the name of this}
\TODO[inline]{Talk about motor/gearbox analog}

\section{Manufacturing, Materials, \& Parts}
Throughout the design, manufacturability and easy assembly was a focus. The entire system is held together with 4 M5 shoulder bots with 6mm diameter shoulders to be used as an accurate bearing surface.

\subsubsection{Bearings}
\label{sec:BearingsAndCalcs}
All 10 bearings used in the design are the same (for simplicity and reduction of cost): 19mm outside diameter x 6mm inside diameter x 6mm thick double shielded ball bearings (Model 626ZZ). Each is rated for \(2.6kN\) dynamic load and \(1.05kN\) static load. Before selecting these bearings, two calculations were required to ensure these bearings could support the forces required.

The first is the requirement of the torsion bar. 

The second force requirement for these bearings were in the knee path cam

\section{Testing \& Results}

\subsection{Material Analysis}

\subsection{Simulation Motion Analysis}

\subsection{Real-world Motion Analysis}